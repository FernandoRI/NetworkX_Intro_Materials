%
%  beamer_template.tex
%
%  Created by Drew Conway on 2010-01-22.
% 
%
\documentclass[xcolor=dvipsnames, 9pt]{beamer}

\newenvironment{code}{\begin{semiverbatim} \begin{footnotesize}}
{\end{footnotesize}\end{semiverbatim}}

\usepackage{graphicx}
\usepackage{amssymb}
\usepackage{amsfonts}
\usepackage{amsmath}
\usepackage{hyperref}
\usepackage{natbib}
\usepackage{color}
\usepackage{pdfsync}
\usepackage{chancery}
\usepackage{movie15}
\usepackage{pgfpages}
\usepackage{fancyvrb}
\usepackage{colortbl}

% \definecolor{white}{rgb}{255,255,255}
% \definecolor{darkred}{rgb}{0.5,0,0}
% \definecolor{darkgreen}{rgb}{0,0.5,0}
% \definecolor{lightblue}{rgb}{0,0,0.7}

% \hypersetup{colorlinks,
%   linkcolor=white,
%   filecolor=darkred,
%   urlcolor=lightblue,
%   citecolor=darkblue}

\usepackage{beamerthemesplit}
\usetheme{Warsaw}
%\usecolortheme[named=Brown]{structure} 
\setbeamertemplate{navigation symbols}{}
\setbeamertemplate{itemize items}[triangle]
\setbeamertemplate{enumerate items}[default]
%\setbeameroption{show notes on second screen}
%\logo{\includegraphics[width = 2cm]{nyulogo.png}}

\newcommand{\R}{\mathbb{R}}
\renewcommand{\d}{\mathsf{d}}
\newcommand{\dd}{\partial}
\newcommand{\E}{\mathsf{E}}
\newcommand{\bb}{\mathbf}

\title{ New Beamer Presentation}
\author{Drew Conway}
\date{\today}

\begin{document} 

\begin{frame}[plain]
  \titlepage  
\end{frame}

\section{Introductions} % (fold)
\label{sec:introductions}

% section introductions (end)

\begin{frame}[plain]
	\frametitle{Introduction}
	I usually have the title and introduction slides in the \texttt{plain} format, and then the rest in \texttt{fragile}
	\begin{itemize}
	   \item I also use a lot of itemized lists
	\end{itemize}
\end{frame}

\subsection{Using columns} % (fold)
\label{sub:using_columns}

\begin{frame}[fragile]
    \frametitle{Sectioning and columns}
    I also like to use sections and subsections to call out where in the talk I am.
    \vspace{3cm}
    \begin{columns}
        \column{0.55\textwidth}
        Columns are also great for side-by-side comparisons
        \column{0.45\textwidth}
        So I use those a lot too
    \end{columns}
\end{frame}

% subsection using_columns (end)

\subsection{Code Examples} % (fold)
\label{sub:code_examples}

\begin{frame}[fragile]
    \frametitle{Template for code example}
    I like to use code blocks to call out examples, and teh \texttt{alert} and \texttt{uncover} functions
        \uncover<1->{In Python the \alert{dict} type is a data structure that represents a key$\rightarrow$value mapping}
        \uncover<2->{\begin{block}{Working with the dict type}}
            \begin{code}
    \uncover<2->{\alert<2>{\# Keys and values can be of any data type}}
    \uncover<2->{>>> fruit\_dict=\{"apple":1,"orange":[0.23,0.11],"banana":True \}} \newline
    \uncover<3->{\alert<3>{\# Can retrieve the keys and values as Python lists (vector)}
    >>> fruit\_dict.keys()
    ["orange","apple","banana"]} \newline
    \uncover<4->{\alert<4>{\# Or create a (key,value) tuple}
    >>> fruit\_dict.items()
    [("orange",[0.23,0.11]),("apple",1),("Banana",True)]}
    \uncover<5->{\alert<5>{\# This becomes especially useful when you master Python ``list comprehension''}}
                \end{code}
            \end{block}
    \uncover<6->{The Python dictionary is an extremely flexible and useful data structure, making it one of the primary advantages of Python over other languages}
\end{frame}

% subsection code_examples (end)

\end{document}